\begin{align}
    \text{Derivata} & & \partial_\mu \equiv \frac{\partial}{\partial x^\mu} := \left (\frac{1}{c}\frac{\partial}{\partial t}, \nabla \right )\\
    & & \partial^\mu \equiv \frac{\partial}{\partial x_\mu} := \left ( \frac{1}{c}\frac{\partial}{\partial t}, -\nabla \right )\\
    \text{Quadrivelocità} & & u^\mu := \frac{dx^\mu}{ds} = \gamma(v)\left (1, \frac{\vec{v}}{c}\right ); \quad u^\mu u_\mu = 1\\
    \text{Intervallo} & & ds = \sqrt{g_{\mu\nu}dx^\mu dx^\nu} = \sqrt{dx_\mu dx^\nu} = \frac{c dt}{\gamma(v)}\\
    \text{Quadriaccelerazione} & & w^\mu := \frac{du^\mu}{ds} = \left ( \frac{\gamma^4}{c^3}\vec{v}\cdot \vec{a}, \frac{\gamma^2}{c^2}a^i + \frac{\gamma^4}{c^4}v^i(\vec{v}\cdot \vec{a}) \right ); \quad w^\mu u_\mu = 0;\\ 
    \text{Quadrimomento} & & p^\mu = mcu^\mu = \left ( \frac{E}{c}, m\gamma(v)v^i \right )\\
    \text{Energia} & & E = \sqrt{m^2c^4 + c^2|\vec{p}|^2} = m\gamma(v) c^2\\
    \text{Quadriforza} & & F^\mu = \frac{dp^\mu}{ds} = \left ( \frac{\gamma}{c^2}\vec{F}\cdot \vec{v}, \frac{\gamma}{c}\vec{F} \right ); \quad F^\mu u_\mu = 0\\
    \beta = \frac{v}{c} = c\frac{p}{E} \Rightarrow \beta = \frac{p}{E} \span \span
\end{align}