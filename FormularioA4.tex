\documentclass[12pt]{article}
\usepackage[usenames, dvipsnames, table]{xcolor}
\usepackage[utf8]{inputenc}
\usepackage[T1]{fontenc}
\usepackage{lmodern}
\usepackage{amsmath}
\usepackage{amsfonts}
\usepackage{comment}
\usepackage{wrapfig}
\usepackage{booktabs}
\usepackage{tikz}
\usepackage{gnuplottex}
\usepackage{epstopdf}
\usepackage{marginnote}
\usepackage{float}
\usetikzlibrary{tikzmark}
\usepackage{graphicx}
\usepackage{cancel}
\usepackage{bm}

\DeclareMathOperator{\sech}{sech}
\DeclareMathOperator{\csch}{csch}
\DeclareMathOperator{\arcsec}{arcsec}
\DeclareMathOperator{\arccot}{arcCot}
\DeclareMathOperator{\arccsc}{arcCsc}
\DeclareMathOperator{\arccosh}{arcCosh}
\DeclareMathOperator{\arcsinh}{arcsinh}
\DeclareMathOperator{\arctanh}{arctanh}
\DeclareMathOperator{\arcsech}{arcsech}
\DeclareMathOperator{\arccsch}{arcCsch}
\DeclareMathOperator{\arccoth}{arcCoth} 

\newif\ifquoteopen
\catcode`\"=\active % lets you define `"` as a macro
\DeclareRobustCommand*{"}{%
   \ifquoteopen
     \quoteopenfalse ''%
   \else
     \quoteopentrue ``%
   \fi
}

\PassOptionsToPackage{table}{xcolor}

\usepackage{soul}

\newcommand{\hlc}[2]{%
  \colorbox{#1!50}{$\displaystyle#2$}}


\usepackage[a4paper,
            total={170mm,257mm},
 left=20mm,
 top=20mm]{geometry}

\newcommand{\q}[1]{``#1''}
\newcommand{\lamb}[2]{\Lambda^{#1}_{\>{#2}}}
\newcommand{\norm}[1]{\left\lVert#1\right\rVert}
\usepackage{fancyhdr}
\pagestyle{fancy}
\fancyhead{} % clear all header fields
\renewcommand{\headrulewidth}{0pt} % no line in header area
\fancyfoot{} % clear all footer fields
\fancyfoot[LE,RO]{\thepage}           % page number in "outer" position of footer line
\fancyfoot[RE,LO]{Francesco Manzali, Marzo 2018} % other info in "inner" position of footer line

\begin{document}
\begin{align}
    \text{Derivata} & & \partial_\mu \equiv \frac{\partial}{\partial x^\mu} := \left (\frac{1}{c}\frac{\partial}{\partial t}, \nabla \right )\\
    & & \partial^\mu \equiv \frac{\partial}{\partial x_\mu} := \left ( \frac{1}{c}\frac{\partial}{\partial t}, -\nabla \right )\\
    \text{Quadrivelocità} & & u^\mu := \frac{dx^\mu}{ds} = \gamma(v)\left (1, \frac{\vec{v}}{c}\right ); \quad u^\mu u_\mu = 1\\
    \text{Intervallo} & & ds = \sqrt{g_{\mu\nu}dx^\mu dx^\nu} = \sqrt{dx_\mu dx^\nu} = \frac{c dt}{\gamma(v)}\\
    \text{Quadriaccelerazione} & & w^\mu := \frac{du^\mu}{ds} = \left ( \frac{\gamma^4}{c^3}\vec{v}\cdot \vec{a}, \frac{\gamma^2}{c^2}a^i + \frac{\gamma^4}{c^4}v^i(\vec{v}\cdot \vec{a}) \right ); \quad w^\mu u_\mu = 0;\\ 
    \text{Quadrimomento} & & p^\mu = mcu^\mu = \left ( \frac{E}{c}, m\gamma(v)v^i \right )\\
    \text{Energia} & & E = \sqrt{m^2c^4 + c^2|\vec{p}|^2} = m\gamma(v) c^2\\
    \text{Quadriforza} & & F^\mu = \frac{dp^\mu}{ds} = \left ( \frac{\gamma}{c^2}\vec{F}\cdot \vec{v}, \frac{\gamma}{c}\vec{F} \right ); \quad F^\mu u_\mu = 0\\
    \beta = \frac{v}{c} = c\frac{p}{E} \Rightarrow \beta = \frac{p}{E} \span \span
\end{align}

\[ %Formuletta nuova [TO DO] Trovare limiti di applicabilità
\beta\gamma = \frac{p}{M}
\]
Infatti $\beta = p/E$, e $\gamma = 1/\sqrt{1-\beta^2}$, perciò:
\[
\beta\gamma = \frac{p}{E} \frac{1}{\sqrt{1-\frac{p^2}{E^2}}} \underset{(*)}{=} \frac{p}{\sqrt{M^2 + p^2}}\frac{1}{\displaystyle \sqrt{\frac{M^2}{M^2+p^2}}} = \frac{p}{\cancel{\sqrt{M^2 + p^2}}}\frac{\cancel{\sqrt{M^2 + p^2}}}{M} = \frac{p}{M}
\]
Dove in $(*)$ si è applicata la relazione di mass-shell: $E = \sqrt{m^2c^4 + p^2c^2}$ con $c = 1$.\\
Relazione tra $\beta$ e $\gamma$:
\[
\gamma = \frac{1}{\sqrt{1-\beta^2}} \Rightarrow \gamma^2 = \frac{1}{1-\beta^2} \Rightarrow (1-\beta^2)\gamma^2 = 1 \Rightarrow \gamma^2 -\beta^2\gamma^2 = 1 \Rightarrow \beta = \frac{\gamma^2 - 1}{\gamma^2}
\]
\end{document}