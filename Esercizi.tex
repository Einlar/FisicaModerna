\documentclass[12pt]{article}
\usepackage[usenames, dvipsnames, table]{xcolor}
\usepackage[utf8]{inputenc}
\usepackage[T1]{fontenc}
\usepackage{lmodern}
\usepackage{amsmath}
\usepackage{amsfonts}
\usepackage{comment}
\usepackage{wrapfig}
\usepackage{booktabs}
\usepackage{tikz}
\usepackage{gnuplottex}
\usepackage{epstopdf}
\usepackage{marginnote}
\usepackage{float}
\usetikzlibrary{tikzmark}
\usepackage{graphicx}
\usepackage{cancel}
\usepackage{bm}
\usepackage{hyperref}
\usepackage{siunitx}

\DeclareMathOperator{\sech}{sech}
\DeclareMathOperator{\csch}{csch}
\DeclareMathOperator{\arcsec}{arcsec}
\DeclareMathOperator{\arccot}{arcCot}
\DeclareMathOperator{\arccsc}{arcCsc}
\DeclareMathOperator{\arccosh}{arcCosh}
\DeclareMathOperator{\arcsinh}{arcsinh}
\DeclareMathOperator{\arctanh}{arctanh}
\DeclareMathOperator{\arcsech}{arcsech}
\DeclareMathOperator{\arccsch}{arcCsch}
\DeclareMathOperator{\arccoth}{arcCoth} 

\newif\ifquoteopen
\catcode`\"=\active % lets you define `"` as a macro
\DeclareRobustCommand*{"}{%
   \ifquoteopen
     \quoteopenfalse ''%
   \else
     \quoteopentrue ``%
   \fi
}

\PassOptionsToPackage{table}{xcolor}

\usepackage{soul}

\newcommand{\hlc}[2]{%
  \colorbox{#1!50}{$\displaystyle#2$}}


\usepackage[a4paper,
            total={170mm,257mm},
 left=20mm,
 top=20mm]{geometry}

\newcommand{\q}[1]{``#1''}
\newcommand{\lamb}[2]{\Lambda^{#1}_{\>{#2}}}
\newcommand{\norm}[1]{\left\lVert#1\right\rVert}
\usepackage{fancyhdr}
\pagestyle{fancy}
\fancyhead{} % clear all header fields
\renewcommand{\headrulewidth}{0pt} % no line in header area
\fancyfoot{} % clear all footer fields
\fancyfoot[LE,RO]{\thepage}           % page number in "outer" position of footer line
\fancyfoot[RE,LO]{Francesco Manzali, Giugno 2018} % other info in "inner" position of footer line

\begin{document}
%Impostazioni booktab (Rimuove quello spazio odioso tra righe)
\setlength{\aboverulesep}{0pt}
\setlength{\belowrulesep}{0pt}
\setlength{\extrarowheight}{.75ex}
\begin{center}
		\line (1,0){350} \\
		[0.25in]
		\huge{\bfseries Fisica moderna}\\
		[2mm]
		\line (1,0){350} \\
		[0.5cm]
		\textsc{\LARGE Esercizi di Relatività}\\
		\textsc{\normalsize Anno accademico 2017-2018}\\ 

	\end{center}

\newgeometry{inner=20mm,
            outer=50mm,% = marginparsep + marginparwidth 
                       %   + 5mm (between marginpar and page border)
            top=20mm,
            bottom=25mm,
            marginparsep=5mm,
            marginparwidth=40mm,
            showframe}
            
\tableofcontents 
\clearpage

\section{19/02/2018}
\subsection{Testo}
Un $\mu^+$ è inizialmente in quiete in un sistema di riferimento inerziale $S$ in un semi-spazio ($\sim \mathbf{R}^2\times \mathbf{R}_+$) ove sono presenti un campo elettrico diretto ortogonalmente al bordo del semi-spazio con modulo $E=10^8$\si{V/m} e un campo magnetico parallelo al bordo, di modulo $B=$$10^8 / \sqrt{2}$\si{V/m}.\\
Il muone è inizialmente ad una distanza $d = 1$m dal bordo. Sotto l'influenza del campo elettromagnetico esso entra al tempo $T$ nell'altro semi-spazio ove il campo magnetico assume lo stesso valore ma il campo elettrico è assente. In tale semi-spazio il muone percorre un arco di circonferenza rientrando poi nel semi-spazio iniziale.\\
Assumendo $m_\mu = 0.1$ GeV/$c^2$ si calcolino:
\begin{enumerate}
    \item il valore $P$ della componente del momento del muone lungo $\vec{E}$ all'istante $T$;
    \item il tempo $T$;
    \item l'energia $W$ del muone all'istante in cui rientra nel primo semi-spazio.
\end{enumerate}
\subsection{Svolgimento}

\section{19/09/2017}
\subsection{Testo}
In un sistema inerziale sono poste ad una distanda $d$ due lastre piane parallele, assunte di spessore nullo e infinitamente estese, perpendicolarmente all'asse $y$. Di esse una è neutra e l'altra ha densità superficiale di carica $\sigma = 10^6$ V/m e densità superficiale di corrente $\mathcal{I}=2\sigma c$ lungo l'asse $x$. Un elettrone è in quiete, equidistante dalle lastre, a $t = 0$.\\
Assumendo come massa per l'elettrone $m=0.5$MeV/$c^2$ si calcolino:
\begin{enumerate}
    \item il valore massimo di $d$ per il quale l'elettrone urta la lastra neutra, muovendosi sotto l'influenza del campo elettromagnetico generato dalla lastra carica;
    \item l'energia che esso possiede all'istante dell'urto, in condizione di $d$ massimo;
    \item l'istante in cui avviene l'urto, in condizione di $d$ massimo.
\end{enumerate}
\textbf{Nota}: possono essere utili le equazioni di Maxwell in forma integrale.
\subsection{Svolgimento}

\section{5/9/2017}
\subsection{Testo}
Un fotone di energia $\mathcal{E}_\gamma = 1.5$GeV urta frontalmente nel laboratorio una particella $f^0$ di uguale energia, dando luogo alla reazione:
\[
\gamma + f^0 \to \rho^0 + \rho^0
\]
Un $\rho^0$ viene emesso in condizioni di angolo massimo, $\theta_m$, rispetto alla direzione del $\gamma$, e decade secondo la reazione:
$\rho^0 \to \pi^+ + \pi^-$
in condizioni di angolo minimo $\phi_m$ tra i pioni.\\
Dopo un tempo $\tau = 10^{-8}$s, misurato nel proprio sistema di riferimento, i due pioni incidono su un rivelatore la cui superficie è ortogonale alla linea di volo del $\rho^0$ che decade.\\
Assumendo che il valore delle masse sia dato in unità di GeV/$c^2$ da: $m_f = 1.3$, $m_\rho = 1.45$, $m_\pi = 0.15$, si determinino:
\begin{enumerate}
    \item l'angolo $\theta_m$;
    \item l'angolo $\phi_m$;
    \item la distanza $d$ tra i punti di impatto dei pioni sul rivelatore nel laboratorio.
\end{enumerate}
\subsection{Svolgimento}
\end{document}