\PassOptionsToPackage{table}{xcolor}
\documentclass[12pt]{article}
\usepackage[utf8]{inputenc}
\usepackage{amsmath}
\usepackage{amsfonts}
\usepackage{comment}
\usepackage{wrapfig}

\usepackage{gnuplottex} %PACCHETTO NECESSARIO
\usepackage{epstopdf} %PACCHETTO NECESSARIO

\usepackage{float}
\usepackage{graphicx}

\usepackage{color}


\begin{document}
\begin{figure}[t]
    \centering%
        \begin{gnuplot}[terminal=epslatex, terminaloptions=color dashed,terminaloptions={size 15cm,9cm}]
            
            #Label sugli assi:
            set xlabel "Tempo $t$ [$\\mu$s]" 
            set ylabel "Tensione $V$ [V]" 

            #Parametri stilistici
            set mxtics 5
            set mytics 2
            
            set key bottom center
            
            set style fill pattern 5 border
            set style line 12 lc rgb '#808080' lt 0 lw 1
            set grid back ls 12
            
            set style line 1 lc "black" lw 1
            set style line 2 lt rgb "#00A000" lw 2 pt 6
            set style line 3 lt rgb "#5060D0" lw 2 pt 2
            set style line 4 lt rgb "#F25900" lw 2 pt 9
            
            #Setta la scala dell'asse x
            set xrange[0:135]
            
            set samples 10000
            
            #Setta parametri della retta di fit da plottare
            a = 1.10646
            b = -0.013524
            
            #Plotta
            plot "Dati/datiRC.txt" u 1:2:3 w yerrorbars lc rgb "blue" title "Dati sperimentali", a+b*x title "Fit lineare" lc rgb "black"
        \end{gnuplot}
        \caption{Fit linearizzato del circuito RC di scarica}
        \label{fit_RC}%
\end{figure}

\end{document}