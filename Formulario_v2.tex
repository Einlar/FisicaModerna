\documentclass{article}
\usepackage{amsmath}
\usepackage{lipsum}
\usepackage{fancyhdr}
\usepackage[usenames, dvipsnames, table]{xcolor}
\usepackage[utf8]{inputenc}
\usepackage[T1]{fontenc}
\usepackage{lmodern}
\usepackage{amsfonts}
\usepackage{comment}
\usepackage{wrapfig}
\usepackage{booktabs}
\usepackage{mathtools}
\usepackage{tikz}
\usepackage{gnuplottex}
\usepackage{epstopdf}
\usepackage{marginnote}
\usepackage{float}
\usetikzlibrary{tikzmark}
\usepackage{graphicx}
\usepackage{cancel}
\usepackage{bm}
\usepackage{hyperref}
\usepackage[a4paper, landscape, left=1.5cm, right=1.5cm, top=1.5cm, bottom=1cm, headsep = 5pt, footskip=0pt, headheight=20pt, marginparwidth=0pt, marginparsep = 0pt]{geometry} %aggiungi showframe per vedere i bordi
\pagestyle{fancy}
\fancyhf{}
\pagenumbering{gobble}
\rhead{A.A. 2017-18}
\chead{\Large \textbf{Formulario di Fisica Moderna}}
\lhead{Francesco Manzali, 1147790}
\fancyfoot{}
\newlength\Colsep
\setlength\Colsep{10pt}
\DeclareMathOperator{\sech}{sech}
\DeclareMathOperator{\csch}{csch}
\DeclareMathOperator{\arcsec}{arcsec}
\DeclareMathOperator{\arccot}{arcCot}
\DeclareMathOperator{\arccsc}{arcCsc}
\DeclareMathOperator{\arccosh}{arcCosh}
\DeclareMathOperator{\arcsinh}{arcsinh}
\DeclareMathOperator{\arctanh}{arctanh}
\DeclareMathOperator{\arcsech}{arcsech}
\DeclareMathOperator{\arccsch}{arcCsch}
\DeclareMathOperator{\arccoth}{arcCoth} 

\begin{document}%
\setlength{\abovedisplayskip}{0pt}%
\setlength{\belowdisplayskip}{0pt}%
\setlength{\abovedisplayshortskip}{0pt}%
\setlength{\belowdisplayshortskip}{0pt}%
\noindent%
\begin{minipage}[t]{.49\textwidth}
\raggedright
{\large\center \textbf{Decadimenti}} $M \to m_1 + m_2$; $\alpha$ = num. part.
\begin{align*}
    {p^*}^2 &= \frac{1}{4M^2}[M^4+(m_1^2-m_2^2)^2-2M^2(m_1^2+m_2^2)]c\\
    \mathcal{E}_1^* &= \frac{M^2+m_1^2-m_2^2}{2M}c^2;\quad \mathcal{E}_2^* = \frac{M^2-m_1^2+m_2^2}{2M}c^2\\
    \beta_M &> \beta_\alpha^* = \frac{p^*}{\mathcal{E}_\alpha^*}\quad \parbox{6cm}{(Emissione in avanti di $\alpha$)} \\
    p_\% &= 100\cdot \frac{1}{2\beta\gamma p^*}\Delta E\quad \parbox{6cm}{(Percentuale di particelle -senza spin- generate ad energia nel range $\Delta E$)}\\
    \mathcal{E}_\alpha &= \gamma_{CM}(\mathcal{E}_\alpha^*+\beta_{CM}c p^*\cos\theta_\alpha^*)\\
    p_{\alpha,x}&=\gamma_{CM}\left (p^*\cos\theta_\alpha^* + \beta_{CM}\frac{\mathcal{E}_\alpha^*}{c} \right ); \quad p_{\alpha,y}=p^*\sin\theta_\alpha^*\\
    \tan\theta_\alpha &= \frac{p_{\alpha,y}}{p_{\alpha,x}}=\frac{\sin\theta_\alpha^*}{\displaystyle \gamma\left (\cos\theta_\alpha^*+\frac{\beta}{\beta_\alpha^*}\right )}\quad \parbox{5cm}{(Angoli $\theta_\alpha$, $\theta_\alpha^*$ sono rispetto a linea di volo)}\\
    \sin\theta_\alpha^{\max}&=\frac{p^*}{m_\alpha\gamma_{CM}\beta_{CM}} = \frac{Mp^*}{m_\alpha p_M} \Rightarrow \cos\theta_\alpha^* = -\frac{\beta_\alpha^*}{\beta}\>\text{(se $\theta_\alpha$ è max)}
\end{align*}
{\center \large \textbf{-Masse uguali}} $M\to m+m$ ($\theta$ è angolo tra particelle)
\begin{align*}
    |p^*_1|&=|p_2^*|=p^*=\sqrt{\frac{M^2}{4}-m^2}; \quad \mathcal{E}_1^* =\mathcal{E}_2^* = \mathcal{E}^* = \frac{M}{2}; \quad \tan\theta(\theta^*) = \frac{A\sin\theta^*}{\sin^2\theta^*+B}\\
    \beta_1^* &= \beta_2^* = \beta^* = \frac{p^*}{\mathcal{E}^*} = \frac{2p^*}{M} = \sqrt{1-\frac{4m^2}{M^2}};\quad  A = \frac{2}{\beta^*\beta\gamma}; \> B= \frac{1}{{\beta^*}^2}-\frac{1}{\beta^2}
\end{align*}
{\center \large \textbf{Urti} (Elastici, bersaglio fermo)} $\mathcal{E}_1 + m_2 \to \mathcal{E}_1' + \mathcal{E}_2'$
\begin{align*}
    {p^*}^2 &= \frac{m_2^2(\mathcal{E}_1^2-m_1^2)}{m_1^2+m_2^2+2m_2\mathcal{E}_1}; \quad \beta_{CM}=\frac{|\vec{p}_1|}{\mathcal{E}_1+m_2}; \quad \sin\theta_{1,max} = \frac{p^*}{c m_1 \beta \gamma}\> (m_1 > m_2)\\
    \mathcal{E}_1' &= \mathcal{E}_1 - \frac{{p^*}^2}{m_2}(1-\cos\theta^*); \quad \mathcal{E}_2' = (\mathcal{E}_1+\mathcal{E}_2)-\mathcal{E}_1'; \quad \sin\theta_{2,max} = \frac{\pi}{2}\\
    \mathcal{E}_1' &= \frac{(\mathcal{E}_1+m_2)(m_1^2+m_2\mathcal{E}_1)+p^2\cos\theta\sqrt{m_2^2-m_1^2\sin^2\theta}}{(\mathcal{E}_1+m_2)^2-p^2\cos^2\theta}\\
    \mathcal{E}_2' &= m_2 \left (\frac{(\mathcal{E}_1+m_2)^2+p^2\cos^2\varphi}{(\mathcal{E}_1+m_2)^2-p^2\cos^2\varphi} \right ) \quad \parbox{4cm}{($\theta$, $\phi$ angoli di particella $1$ e $2$ rispetto a direz. di volo)}
\end{align*}
{\center \large -\textbf{Collisione frontale} (stesse masse)} $m_1 + m_1 \to m_2' + m_2'$
\begin{align*}
    \mathcal{E}_1^* &= \mathcal{E}_2^* = \mathcal{E}_1'^* = \mathcal{E}_2'^* \quad (= m_2) \quad \parbox{4cm}{se si ha la produzione in soglia}\\
    \mathcal{E}_1 &= \gamma(\mathcal{E}^* + \beta p^*); \quad p_1 = \gamma(\beta \mathcal{E}^* + p^*); \quad \beta_{CM} = \frac{p_{tot}}{\mathcal{E}_{tot}}\\
    \mathcal{E}_2 &= \gamma(\mathcal{E}^* - \beta p^*); \quad p_2 = \gamma(\beta \mathcal{E}^* - p^*)
\end{align*}
\raggedright
\end{minipage}%
\begin{minipage}[t]{.5\textwidth}
\vspace{30pt}
\hspace{-10pt}
\begin{gnuplot}[terminal=epslatex, terminaloptions=color dashed, terminaloptions={size 12.5cm,17.2cm}]
            
            set ylabel " " 
            
            set xrange [0:pi]
            
            
            set mxtics 5
            set mytics 2
            
            
            set xzeroaxis
            
            set xtics format " "
            set xtics 1
            
            set key center bottom
            set key reverse
            
            set style fill pattern 5 border
            set style line 12 lc rgb '#808080' lt 0 lw 1
            set grid back ls 12
            
            set style line 1 lc "blue" lw 2
            set style line 2 lt rgb "#black" lw 1
            set style line 3 lt rgb "#5060D0" lw 2 pt 2
            set style line 4 lt rgb "red" lw 2 pt 9
            
            
            set tmargin 0
            set bmargin 0
            set lmargin 2
            set rmargin 1
            
            beta = 0.5
            betastar = 0.5
            gamma(beta) = 1/sqrt(1-beta*beta)
            A(beta, betastar) = 2/(gamma(beta)*betastar*beta)
            B(beta, betastar) = 1/(betastar*betastar) - 1/(beta*beta)
            f(x) = A(beta,betastar)*sin(x)/(B(beta,betastar)+sin(x)*sin(x))
            
            set multiplot layout 4,2 margins 0.05,0.95,.1,.99 spacing 0,0
            
            #B < -1
            set xtics format " "
            set xtics (0, pi/4, pi/2, 3*pi/4, pi)
            set x2tics ("$0$" 0,"$\\displaystyle\\frac{\\pi}{4}$" pi/4,"$\\displaystyle\\frac{\\pi}{2}$" pi/2, "$\\displaystyle\\frac{3\\pi}{4}$" 3*pi/4, "$\\pi$" pi) offset 0, 0.3
            set x2label "Angolo $\\theta^*$" offset 0, 0.5
            
            set ytics -3.5,0.5,-0.5
            set yrange [-4:0]
            
            beta = 0.4
            betastar = 0.5
            set ylabel "$B < -1$" offset 2, 0
            
            set label 1 at pi/2, -0.3 "$\\beta = 0.4;\\>\\beta^* = 0.5; \\> B = -2.25$" center front
            
            plot f(x) title "$\\>\\tan\\theta(\\theta^*)$" ls 1
            
            unset label 1
            unset x2label
            
            set ytics format " "
            unset x2tics
            
            set x2label "Circonferenza goniometrica"
            set parametric
            set key center top
            set xrange [-1.1:1.1]
            set xtics (-1, 0, 1)
            
            set ytics (0, sqrt(2)/2, 1)
            set y2tics ("$0$" 0, " " sqrt(2)/2, "$1$" 1)
            
            set trange [0:pi]
            set yrange [0:1.7]
            unset ylabel
            
            set label 2 at 0.6, 0.16 "$\\displaystyle 2\\arctan \\frac{\\beta^*}{\\gamma\\beta}$" center front
            set label 3 at -sqrt(2)/2, sqrt(2)/2 + 0.1 "$\\bar{\\theta}$" center front
            set label 4 at 0, 1.5"$\\displaystyle\\bar{\\theta}^* = \\frac{\\pi}{2}$" center front
            
            fx(t) = cos(t)
            fy(t) = sin(t)
            plot fx(t),fy(t) ls 2 notitle, fx(t/4+3*pi/4),fy(t/4+3*pi/4) lc rgb "red" lw 4 notitle, -t,t notitle,  0.2*fx(t/4*3),0.2*fy(t/4*3) ls 3 notitle
            
            unset label 2
            unset label 3
            unset label 4
            unset x2tics
            unset y2tics
            
            unset parametric
            unset yrange
            unset x2label
            
            #-1 < B < 0
            set xtics (0, pi/4, pi/2, 3*pi/4, pi)
            set xrange [0:pi]
            beta = 0.47
            betastar = 0.5
            set yrange[-150:150]
            set ytics (-100, -50, 0, 50, 100)
            set ytics format "%.0s"
            set key center bottom
            set ylabel "$-1 < B < 0$" offset 2, 0
            
            set label 1 at pi/2, 130 "$\\beta = 0.47;\\>\\beta^* = 0.5; \\> B = -0.53$" center front
            
            plot f(x) title "$\\>\\tan\\theta(\\theta^*)$" ls 1
            
            unset ylabel
            unset label 1
            
            set parametric
            set key center top
            set xrange [-1.1:1.1]
            set xtics (-1, 0, 1)
            set ytics (0, sqrt(2)/2, 1)
            set trange [0:pi]
            set yrange [0:1.7]
            set ytics format " "
            set y2tics ("$0$" 0, " " sqrt(2)/2, "$1$" 1)
            unset ylabel
            
            set label 2 at 0.6, 0.16 "$\\displaystyle 2\\arctan \\frac{\\beta^*}{\\gamma\\beta}$" center front
            set label 3 at sqrt(2)/2, sqrt(2)/2 + 0.1 "$\\bar{\\theta}$" center front
            set label 4 at 0, 1.5"$\\displaystyle\\bar{\\theta}^* = \\frac{\\pi}{2}$" center front
           
            plot fx(t),fy(t) ls 2 notitle, fx(t*3/4+pi/4),fy(t*3/4+pi/4) lc rgb "red" lw 4 notitle, t,t notitle, 0.2*fx(t/4),0.2*fy(t/4) ls 3 notitle
            
            unset parametric
            unset label 2
            unset label 3
            unset label 4
            unset y2tics
            
            #0<B<1
            set xtics (0, pi/4, pi/2, 3*pi/4, pi)
            set xrange [0:pi]
            beta = 0.53
            betastar = 0.5
            set key center bottom
            set yrange [0:4]
            set ytics format "%.1t"
            set ytics (0.5, 1, 1.5, 2, 2.5, 3, 3.5)
            set ylabel "$0 < B < 1$" offset 1, 0
            
            set label 1 at pi/2, 3.7 "$\\beta = 0.53;\\>\\beta^* = 0.5; \\> B = 0.39$" center front
        
            plot f(x) title "$\\>\\tan\\theta(\\theta^*)$" ls 1
            unset label 1    
            
            set parametric
            set key center top
            set xrange [-1.1:1.1]
            set xtics (-1, 0, 1)
            set ytics (0, sqrt(3)/2, 1)
            set trange [0:pi]
            set yrange [0:1.7]
            set ytics format " "
            unset ylabel
            set y2tics ("$0$" 0, " " sqrt(2)/2, "$1$" 1)
            
            set label 2 at 0.2, 0.36 "$\\displaystyle \\arctan\\frac{1}{\\gamma\\sqrt{\\beta^2-\\beta^{*^2}}}$" center front
            set label 3 at 0.5, sqrt(3)/2 + 0.1 "$\\theta_0$" center front
            set label 4 at 0, 1.4"$\\displaystyle\\theta^*_0 = \\arcsin\\left (\\sqrt{\\frac{1}{{\\beta^*}^2}-\\frac{1}{\\beta_{CM}^2}} \\right) $" center front
            
            plot fx(t),fy(t) ls 2 notitle, fx(t/3),fy(t/3) lc rgb "red" lw 4 notitle, t,sqrt(3)*t notitle, 0.2*fx(t/3),0.2*fy(t/3) ls 3 notitle 
            
            unset parametric
            unset label 2
            unset label 3
            unset label 4
            unset y2tics
            
            # B > 1
            set xlabel "Angolo $\\theta^*$" offset 0, -0.5
            set ylabel "$B > 1$" offset 2, 0
            set xtics format "%.0s"
            set xtics ("$0$" 0,"$\\displaystyle\\frac{\\pi}{4}$" pi/4,"$\\displaystyle\\frac{\\pi}{2}$" pi/2, "$\\displaystyle\\frac{3\\pi}{4}$" 3*pi/4, "$\\pi$" pi) offset 0, -0.25
            set xrange [0:pi]
            set yrange[0:1.8]
            set ytics 0.25, 0.25, 1.5
            set ytics format "%.2t"
            
            set key center bottom
            
            beta = 0.71
            
            set label 1 at pi/2, 1.70 "$\\beta = 0.71;\\>\\beta^* = 0.5; \\> B = 2$" center front
            
            plot f(x) title "$\\>\\tan\\theta(\\theta^*)$" ls 1
            
            set xtics format " "
            unset xtics
            unset xlabel
            unset label 1
            unset ylabel
            
            set parametric
            set key center top
            set xrange [-1.1:1.1]
            set xtics (-1, 0, 1) offset 0.1, 0
            set ytics (0, sqrt(2)/2, 1)
            set trange [0:pi]
            set yrange [0:1.7]
            set ytics format " "
            set xtics format "%.0t"
            set y2tics ("$0$" 0, " " sqrt(2)/2, "$1$" 1)
            unset ylabel
            set xlabel "Circonferenza goniometrica"
            
            set label 2 at 0.6, 0.16 "$\\displaystyle 2\\arctan \\frac{\\beta^*}{\\gamma\\beta}$" center front
            set label 3 at sqrt(2)/2, sqrt(2)/2 + 0.1 "$\\bar{\\theta}$" center front
            set label 4 at 0, 1.5"$\\displaystyle\\bar{\\theta}^* = \\frac{\\pi}{2}$" center front
           
            plot fx(t),fy(t) ls 2 notitle, fx(t/4),fy(t/4) lc rgb "red" lw 4 notitle, t,t notitle, 0.2*fx(t/4),0.2*fy(t/4) ls 3 notitle
            
            unset parametric
            unset label 2
            unset label 3
            unset label 4
            unset y2tics
            
            
            unset multiplot
        \end{gnuplot}
\end{minipage}%%
\clearpage%
\noindent%
\begin{minipage}[t]{.25\textwidth}
\raggedright
{\Large \textbf{Dinamica}}
\begin{align*}
    x^\mu &= (ct, x, y, z)\\
    ds &= \sqrt{dx_\mu dx^\nu} = \frac{c\,dt}{\gamma(v)}\\
    \beta &= \frac{v}{c}; \>\> \gamma = \frac{1}{\sqrt{1-\beta^2}}\\
    u^\mu &= \frac{dx^\mu}{ds} = \gamma(v)\left (1, \frac{\vec{v}}{c} \right )\\
    w^\mu &= \frac{du^\mu}{ds} = \left (\frac{\gamma^4}{c^3}\vec{v}\cdot \vec{a}, \frac{\gamma^2}{c^2}a^i+\frac{\gamma^4}{c^4}v^i\vec{v}\cdot \vec{a} \right )\\
    p^\mu &= mcu^\mu = \left (\frac{E}{c}, m\gamma(v)v^i \right )\\
    \mathcal{E} &= \sqrt{m^2c^4 + c^2|\vec{p}|^2} = m\gamma(v)c^2\\
    \mathcal{F}^\mu &= \frac{dp^\mu}{ds} = \left (\frac{\gamma}{c^2}\vec{F}\cdot \vec{v}, \frac{\gamma}{c}\vec{F} \right )\\
    \frac{dp^\mu}{ds} &= \frac{q}{c}F^{\mu\nu}u_\nu\\
    u^\mu u_\mu &= 1; \quad w_\mu u^\mu = 0\\
    \begin{dcases}
    \frac{d\vec{p}}{ds} = \frac{\gamma}{c}q(\vec{E}+\frac{\vec{v}}{c}\times\vec{B})\\
    \frac{d\mathcal{E}}{dt} = q\vec{E}\cdot \vec{v}
    \end{dcases}
    \span
\end{align*} 
\textbullet\,{\large \textbf{Invarianti}}
\begin{align*}
    p^\mu p_\mu &= m^2 c^2\\
    \vec{E}\cdot \vec{B}; \quad E^2-B^2\span
\end{align*}
{\center \textbf{-Urti}}
\begin{align*}
    p_1 + p_2 &\to p_3 + p_4\\
    W^2 = s &= (p_1+p_2)^2 = (p_3+p_4)^2\\
    t &= (p_1-p_3)^2 = (p_4-p_2)^2\\
    u &= (p_1-p_4)^2 = (p_3-p_2)^2
\end{align*}
\textbullet\,{\large \textbf{Formule generali}}\\ %Formule decadimenti qui
\begin{align*}
    N(t) &= N_0 \exp{-t/\tau}\\
    \tau &= 1/\lambda; \quad t_{\frac{1}{2}} = \ln(2\tau)
\end{align*}
($\tau$ = tempo di decadimento, misurato nel sdr in cui la particella è in \textbf{quiete}. $\lambda$ = cost. di decad., $t_{1/2}$ = tempo di dimezzamento)
\end{minipage}%
\begin{minipage}[t]{.25\textwidth}
{\Large \textbf{Relazioni utili}}
\begin{align*}
    \beta &= \frac{v}{c} = c\frac{p}{E} \Rightarrow \beta = \frac{p}{E};\> \gamma = \frac{1}{\sqrt{1-\beta^2}}\\
    \Rightarrow \beta \span= \frac{\gamma^2-1}{\gamma^2}\\
    \beta\gamma &= A = \frac{p}{M} \Rightarrow \beta = \frac{A}{\sqrt{1+A^2}}; \> \gamma = \sqrt{A^2+1}\\
    \mathcal{E} &= \sqrt{m^2c^4+c^2|\vec{p}|^2}
\end{align*}
\vspace{10pt}
{\Large \textbf{Equazioni di Maxwell}}
\begin{align*}
    \vec{\nabla}\cdot \vec{E} &= 4\pi\rho  &\vec{\nabla}\times\vec{E} &= -\frac{1}{c}\frac{\partial}{\partial t}\vec{B}\\
    \vec{\nabla}\cdot\vec{B} &= 0  &\vec{\nabla}\times\vec{B} &= \frac{4\pi}{c}\vec{J}+\frac{1}{c}\frac{\partial}{\partial t}\vec{E}\\
    \vec{F} =  q\left (\vec{E}+\frac{\vec{v}}{c}\times\vec{B}\right )\span\span\span \\
    \int_\Sigma \vec{E}\cdot d\vec{\Sigma} &= 4\pi\int_\tau \rho d\tau \span\span\span\\
    \oint_\gamma \vec{B}\cdot \vec{s} = \frac{1}{c}\left (4\pi \int_\Sigma \vec{J}\cdot d\vec{\Sigma} + \frac{\partial}{\partial t}\int_\Sigma \vec{E}\cdot d\vec{\Sigma} \right )\span\span\span
\end{align*}
\vspace{10pt}
{\Large \textbf{Composizione velocità}}
\[
\begin{dcases}
v_x' = \frac{v_x-V}{1-\frac{V v_x}{c^2}}\\
v_y' = v_y \frac{\sqrt{1-\frac{V^2}{c^2}}}{1-\frac{V v_x}{c^2}}\\
v_z' = v_z \frac{\sqrt{1-\frac{V^2}{c^2}}}{1-\frac{V v_x}{c^2}}
\end{dcases}
\]
\raggedright
\end{minipage}%
\begin{minipage}[t]{.5\textwidth}
{\Large \textbf{Trasformazioni}} (boost lungo $\hat{x}$)
\begin{align*}
    x &= \gamma(x'+vt) & x' &= \gamma(x-vt)\\
    t &= \gamma\left (t' + \frac{v}{c^2}x'\right ) &t' &= \gamma\left (t-\frac{v}{c^2}x \right )\\
    \span\span\span
    x' = \begin{bmatrix}
    \gamma & -\beta\gamma & 0 & 0\\
    -\beta\gamma & \gamma & 0 & 0\\
    0 & 0 & 1 & 0\\
    0 & 0 & 0 & 1
    \end{bmatrix}x; \quad x = \begin{bmatrix}
    \gamma & \beta\gamma & 0 & 0\\
    \beta\gamma & \gamma & 0 & 0\\
    0 & 0 & 1 & 0\\
    0 & 0 & 0 & 1
    \end{bmatrix} x'\\
    \span\span\span
    \begin{cases}
    E_x' = E_x\\
    E_y' = \gamma(E_y - \beta B_z)\\
    E_z' = \gamma(E_z + \beta B_y)
    \end{cases} \quad \begin{cases}
    B_x' = B_x\\
    B_y' = \gamma(B_y + \beta E_z)\\
    B_z' = \gamma(B_z - \beta E_y)
    \end{cases}
\end{align*}
Se $\vec{E}\cdot\vec{B} = 0$ (campi $\perp$) $\to$ si può annullare il campo con modulo minore.\\
Se $E < B$, allora $E' = 0$, $B' = \sqrt{B^2-E^2} = B/\gamma$, $\beta = E/B$\\
Se $B < E$, allora $E' = \sqrt{E^2-B^2} = E/\gamma$, $B' = 0$, $\beta = B/E$.\\
(\textbf{Nota}: vanno trasformate anche le condizioni iniziali!)\\
{\Large \textbf{Moto in campo $\vec{E}$}} (Campo lungo $\hat{x}$)
\begin{align*}
    \frac{d\vec{p}}{dt}&=q\vec{E}; \quad \frac{d\mathcal{E}}{dt} = q(\vec{E}\cdot \vec{v}) \Rightarrow \Delta\mathcal{E} = qE\Delta x \quad \parbox{5cm}{(Vale anche in presenza di $\vec{B}$ costante)}\\
    \span
    \begin{cases}
    p_x(t) = qEt\\
    p_y(t) = p_{0y}\\
    p_z(t) = 0
    \end{cases} \quad v_x(t) = \frac{c^2 (qEt)}{\displaystyle \mathcal{E}_0 \left [
    1+ \left(\frac{cqEt}{\mathcal{E}_0}\right )^2
    \right ]^{1/2}}\\
    \span
    \begin{dcases}
    x(t) = x_0 + \frac{1}{\alpha}(\sqrt{1+(\alpha c t)^2} -1)\\
    y(t) = y_0 + \frac{p_{0y}c}{qE}\arcsinh(\alpha c t)\\
    z(t) = z_0
    \end{dcases} \text{con} \> \alpha = \frac{qE}{\mathcal{E}_0} \quad \parbox{5cm}{(Misurate tutte nel sdr in cui si misurano le distanze!)}\\
    x(t) \span= \frac{\mathcal{E}_0}{E}\left (\cosh \left (\frac{qE y(t)}{p_{0y}c} \right )-1 \right )
\end{align*}
{\Large \textbf{Moto in campo $\vec{B}$}}
\begin{align*}
    \omega &= \frac{qcB}{\mathcal{E}} = \frac{qB}{m\gamma(v)c} \quad \parbox{5cm}{(Grandezze rispetto allo stesso sdr!)}\\
    R &= \frac{|v_\perp|}{\omega} = \frac{|v_\perp|m\gamma(v)c}{qB} = \frac{c|p_\perp|}{qB} \> \parbox{4cm}{(Nota: proiezioni su assi diversi da quello di boost non variano)}
\end{align*}
\raggedright
\end{minipage}%
\end{document}